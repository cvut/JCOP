\documentclass[]{article} 

\usepackage[english]{babel}
\usepackage[utf8]{inputenc}
\usepackage{cmap}
\usepackage[T1]{fontenc}
\usepackage{multirow}
\usepackage{graphicx}
\usepackage{setspace}
\usepackage{url}

\oddsidemargin=-5mm
\evensidemargin=-5mm
\marginparwidth=.08in
\marginparsep=.01in
\marginparpush=5pt
\topmargin=-15mm
\headheight=12pt
\headsep=25pt
%\footheight=12pt
\footskip=30pt
\textheight=25cm
\textwidth=17cm
\columnsep=2mm
\columnseprule=1pt
\parindent=15pt
\parskip=2pt

\begin{document}
\begin{center}
\bf Diploma thesis outline:\\[5mm]
    Java Combinatorial Optimization Package (JCOP)\\[5mm] 
       Ondřej Skalička\\[2mm]
5th year, FEL ČVUT, Karlovo nám. 13, 121 35 Praha 2\\[2mm]
\today
\end{center}

\section{Introduction}


\subsection{Motivation}

Why to create JCOP (missing a tool for easy comparison of different algorithms which allows simple problem-algorithm communication; education at FEL.CVUT)

\subsection{Purpose}

What should and should not be JCOP used for (benchmarking - comparing different algs)

\subsection{State of the Art}

Other similar projects (JCOOL, OAT, FakeGame...)

\section{Combinatorial Problems}

Where are they used, what for, theoretical importance , P/NP/NP-C, common problems (SAT, TSP, Knapsack...)

\subsection{Algorithms}

DFS/BFS (+other graph search based), local non-graph based (SA, Tabu), global search (genetics and variations)

\section{Analysis}

\subsection{Use cases}

What can different users (developer/student, teacher, ...) do

\subsection{Functional/Nonfunctional/Other requirements}

Exceptations what JCOP should be able to do, platform (why java) etc. (+mention tutorials)

\section{Implementation}

\subsection{Technologies}

Java, SVN, SourceForge, Enterprise Architect

\subsection{JCOP}

Main parts of platform, what is responsible for what

\subsection{Implementation details}

Core of thesis, details (text, source code examples, model/diagram screenshots) about all parts of JCOP, why, where, how about them

\subsection{Adding new elements}
\label{sec:adding-new-elements}

How to add new algorithms/problems/conditions/renders/solvers

\subsection{Tests}

Unit tests (TestNG)

\section{Experimental results}

\subsection{Expected/Real results}

Expectations such as "genetics performs poorly on SAT" etc, benchmark these known result and confirm that JCOP works ok

\section{Conclusion}

\subsection{Future work}

Possible extensions (GUI, adding new elements as in \ref{sec:adding-new-elements}, distributed execution..)

\subsection{references}

\subsection{appendixes}

Much like appendixes online, lists of all implemented problems/solver/algorithms, used utils/libraries etc.


%\section{Literatura}
%
%\begin{thebibliography}{1}
%\bibitem{web1}{\em Stránky předmětu X36PAR na Service} \\
%               \url{http://service.felk.cvut.cz/courses/X36PAR/}
%\end{thebibliography}

\end{document}








